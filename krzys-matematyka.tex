\documentclass[11pt]{article}
\usepackage[utf8]{inputenc}
\usepackage{polski} 
\usepackage{geometry}
\geometry{a4paper}
\usepackage{graphicx}
\usepackage{booktabs}
\usepackage{array}
\usepackage{paralist}
\usepackage{verbatim}
\usepackage{subfig}
\usepackage{amsmath}
\usepackage{float}
\usepackage{amsthm}
\usepackage{amssymb}
\usepackage{amsfonts}
\usepackage{thmtools}
\usepackage[inline]{enumitem}
\theoremstyle{definition}
\newtheorem{zad}{Zadanie}
\newtheorem{theorem}{Twierdzenie}
\newtheorem{definition}{Definicja}
\newtheorem{lemma}{Lemat}
\renewcommand*{\proofname}{Rozwiązanie}

\usepackage{fancyhdr}
\pagestyle{fancy}
\renewcommand{\headrulewidth}{0pt}
\usepackage{sectsty}
\allsectionsfont{\sffamily\mdseries\upshape}
\usepackage[nottoc,notlof,notlot]{tocbibind}
\usepackage[titles,subfigure]{tocloft}
\renewcommand{\cftsecfont}{\rmfamily\mdseries\upshape}
\renewcommand{\cftsecpagefont}{\rmfamily\mdseries\upshape}

\title{Matematyka klasa 5}

\begin{document}
\maketitle

\section{Mnożenie pod kreską}

\begin{equation*}\begin{array}{c}
\phantom{\times00}384\\
\underline{\times\phantom{000}56}\\
\phantom{\times0}2304\\
\underline{+1920\phantom9}\\
\phantom\times21504
\end{array}\end{equation*}

\begin{minipage}[t]{0.3\linewidth}\flushright
\begin{equation*}\begin{array}{c}
\phantom{\times0000}9876\\
\underline{\times\phantom{0000}1234}\\
\phantom{+000}39504\\
\phantom{+00}29628\phantom0\\
\phantom{+0}19752\phantom0\phantom0\\
\underline{+\phantom{0}9876\phantom0\phantom0\phantom0}\\
\phantom{+}12186984
\end{array}\end{equation*}
\end{minipage}
\begin{minipage}[t]{0.1\linewidth}\flushleft\begin{align*}
2 \\
3 \\
3 \\
3 \\
1 \\
2 \\
2
\end{align*}\end{minipage}
\begin{minipage}[t]{0.1\linewidth}\flushleft\begin{align*}
2 \\
1 \\
1 \\
1 \\
1
\end{align*}\end{minipage}



%\section{Trójkąt równoboczny}

%
%\section{Zadania ABCD}
%\begin{zad}
%Które z poniższych liczb będą podzielne przez 3?
%
%  \begin{enumerate*}[label=(\Alph*)]
%    \item 123456789
%    \item 999999991
%    \item 100000002
%    \item 123123...123
%  \end{enumerate*}
%\end{zad}
%
%\begin{zad}
%Jeden bok prostokąta ma 9 cm, a drugi jest od niego o 2 cm krótszy. Obwód tego prostokąta wynosi:
%
%  \begin{enumerate*}[ label=(\Alph*)]
%\item 32 cm
%\item 20 cm
%\item 40 cm
%\item 16 cm
%\end{enumerate*}
%\end{zad}
%
%\begin{zad}
%Które warunki są konieczne aby skonstruować kwadrat?
%
%\begin{enumerate}[label=(\Alph*)]
%\item figura o czterech bokach jednakowej długości
%\item figura o czterech kątach jednakowej miary
%\item figura o czterech wierzchołkach w jednakowym miejscu
%\item figura o czterech przekątnych przecinających się pod kątem $90^\circ$
%\end{enumerate}
%\end{zad}
%
%\begin{zad}
%Znajdź prawdziwe zdania:
%
%\begin{enumerate}[label=(\Alph*)]
%\item Każdy trapez jest równoległobokiem
%\item Każdy równoległobok jest trapezem
%\item Każdy romb jest kwadratem
%\item Każdy kwadrat jest rombem
%\item Każdy prostokąt jest kwadratem
%\item Każdy kwadrat jest prostokątem
%\end{enumerate}
%\end{zad}
%
%\begin{zad}
%Pole kwadratu o boku 5 cm wynosi:
%\begin{enumerate}[label=(\Alph*)]
%\item 25 cm
%\item 25 cm$^2$
%\item 25 cm$^3$
%\end{enumerate}
%\end{zad}
%
%\begin{zad}
%Adam zmierzył bok kwadratu i uzyskał wartość 50 cm. Wywnioskował z tego, że pole kwadratu musi być równe $2500$ cm$^2$. Z drugiej strony, Bob zmierzył bok tego samego kwadratu i uzyskał wartość 0.5 m. Wywnioskował zatem, że kwadrat musi mieć pole 0.25 m$^2$. Następnie Cyprian dokonał pomiaru tego samego kwadratu uzyskując długość boku równą 500 mm. Na podstawie tego wywnioskował, że kwadrat ma pole 1000 mm$^2$. Który z nich ma rację?
%
%\begin{enumerate}[label=(\Alph*)]
%\item Adam
%\item Bob
%\item Cyprian
%\item Nikt z powyższych
%\end{enumerate}
%\end{zad}
%
%
%\section{Geogebra}
%
%\begin{zad}
%Skonstruuj trójkąt prostokątny o długościach przyprostokątnych równych 3 oraz 4. Jaka będzie długość przeciwprostokątnej?
%\end{zad}
%
%\begin{zad}
%Skonstruuj kwadrat o boku 1. Sprawdź, jaka będzie długość jego przekątnej.
%\end{zad}
%
%\begin{zad}
%Skonstruuj kwadrat oparty o przekątną kwadratu z poprzedniego zadania.
%\end{zad}
%
%\begin{zad}
%Skonstruuj romb. Warunek - jego mniejszy kąt ma być równy $30^\circ$.
%\end{zad}
%
%

\end{document}
