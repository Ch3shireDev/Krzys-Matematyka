\documentclass[12pt]{article}
%
%\usepackage{background}
%
%\newlength\mylen
%\setlength\mylen{\dimexpr\paperwidth/40\relax}
%
%\SetBgScale{1}
%\SetBgAngle{0}
%\SetBgColor{blue!30}
%\SetBgContents{\tikz{\draw[step=\mylen] (-.5\paperwidth,-.5\paperheight) grid (.5\paperwidth,.5\paperheight);}}

\usepackage[scaled=10]{helvet}

\usepackage[utf8]{inputenc}
\usepackage{polski} 
\usepackage{geometry}
\geometry{a4paper}
\usepackage{graphicx}
\usepackage{booktabs}
\usepackage{array}
\usepackage{paralist}
\usepackage{verbatim}
\usepackage{subfig}
\usepackage{amsmath}
\usepackage{float}
\usepackage{amsthm}
\usepackage{amssymb}
\usepackage{amsfonts}
\usepackage{thmtools}
\usepackage[inline]{enumitem}
\theoremstyle{definition}
\newtheorem{zad}{Zadanie}
\newtheorem{theorem}{Twierdzenie}
\newtheorem{definition}{Definicja}
\newtheorem{lemma}{Lemat}
\renewcommand*{\proofname}{Rozwiązanie}

\usepackage{fancyhdr}
\pagestyle{fancy}
\renewcommand{\headrulewidth}{0pt}
\usepackage{sectsty}
\allsectionsfont{\sffamily\mdseries\upshape}
\usepackage[nottoc,notlof,notlot]{tocbibind}
\usepackage[titles,subfigure]{tocloft}
\renewcommand{\cftsecfont}{\rmfamily\mdseries\upshape}
\renewcommand{\cftsecpagefont}{\rmfamily\mdseries\upshape}

%\DeclareMathSizes{20}{20}{20}{20}

\title{Matematyka klasa 5}

\begin{document}
%\maketitle

%\section{Arytmetyka elementarna}
%\subsection{Mnożenie pisemne}
%
%\begin{equation*}\begin{array}{c}
%\phantom{\times00}384\\
%\underline{\times\phantom{000}56}\\
%\phantom{\times0}2304\\
%\underline{+1920\phantom9}\\
%\phantom\times21504
%\end{array}\end{equation*}

%	\subsection{Dzielenie pisemne}
%	
%	 921 : 4 = 23 r 1
%	
%	\begin{equation*}
%	\begin{array}{l}
%	\underline{\phantom{-0}23}\\
%	\phantom{-}921\quad:\quad 4\\
%	\underline{-8\phantom{0}}\\
%	\phantom{-}12\\
%	\underline{-12\phantom{0}}\\
%	\phantom{-00}1
%	\end{array}
%	\end{equation*}
%	

\begin{minipage}[t]{0.3\linewidth}\flushright
\begin{equation*}\begin{array}{c}
\phantom{\times0000}9876\\
\underline{\times\phantom{0000}1234}\\
\phantom{+000}39504\\
\phantom{+00}29628\phantom0\\
\phantom{+0}19752\phantom0\phantom0\\
\underline{+\phantom{0}9876\phantom0\phantom0\phantom0}\\
\phantom{+}12186984
\end{array}\end{equation*}
\end{minipage}
\begin{minipage}[t]{0.1\linewidth}\flushleft\begin{align*}
2 \\
3 \\
3 \\
3 \\
1 \\
2 \\
2
\end{align*}\end{minipage}
\begin{minipage}[t]{0.1\linewidth}\flushleft\begin{align*}
2 \\
1 \\
1 \\
1 \\
1
\end{align*}\end{minipage}




\end{document}